\documentclass{ki019}

%\pagestyle{plain}

\begin{document}

\section{Organisation du KI et ses postes}

\freesection{Introduction}

Ce document recense et décrit les postes à pourvoir pour le KI 020. Bien sûr ce n’est qu’une ébauche d’organisation, et le KI 020 pourra lui même décider de son organisation au cours de l’année si le besoin s’en fait ressentir.

\sectionenum{Le bureau}

\subsectionenum{Prez'}

\emph{Le Prez’} forme avec le Prez’ Tech’ le noyau du KI. Il coordonne, organise et supervise toutes les activités du KI. De l’organisation de l’amphi de présentation au début de l’année, à l’écriture du discours de passation, en passant par l’organisation des centrales d’achat, le back-up des formations, et la médiations entre les différents acteurs du KI, il devra faire preuve de patience, d’implication, d’un sens pointu de l’organisation et entretenir de bonnes relations (pacifiques si possible) avec l’ensemble de l’école. Le tout avec le sourire, of course !

\subsectionenum{Prez' Tech}

\emph{Le Prez’ Tech} forme avec le Prez’ le noyau du KI. Il est chargé d’assurer le fonctionnement des multiples aspects techniques du KI, des plus critiques comme le réseau Internet des résidences aux aspects moins importants comme la qualité des sheets dans le Drive. C’est un poste qui nécessite une compréhension globale du monde informatique : hardware, software, développement, réseaux, ... Il doit pouvoir discuter avec les Respos sur tous les aspects techniques afin de pouvoir diriger les efforts, répartir les tâches et mettre la bonne ambiance !

\subsectionenum{Vice Prez'}

\emph{Le Vice Prez’} est le bras droit du Prez’. Prez’, c’est un boulot sympathique mais parfois lourd, surtout quand la LAN tombe au même moment que la formation Linux, les partiels, une réunion avec l’administration ... Comme il n’y a pas 35h par jour, le Vice-Prez’ est là pour alléger sa charge de travail ! Aucune compétence technique requise, c’est un poste assez souple, selon les années et l’équipe du KI : il peut se voir attribuer la gestion de la modération, avoir/développer des compétences techniques et aider le Prez’ Tech, ... Bref, le Vice-Prez’, c’est celui qui est là pour aider tout le monde et apporter un peu d’amour dans ce monde de brute !

\subsectionenum{Sec'Gen'}

\emph{Le Sec’Gen’} est en quelques sortes le scribe du KI. Sa mission principale est de prendre des notes pendant les réunions et d’écrire de très jolis rendus LaTeX pour tenir l’ensemble du club au courant des décisions qui sont prises par le bureau, et des évènements à venir. Être Sec’Gen’ implique donc d’être un peu au courant de ce qui se trame au KI, et donc d’être présent et actif sur la plupart des canaux de communication. Pour résumer le Sec’Gen’ est la plume et l’oreille du KI

\subsectionenum{Trez'}

\emph{Le Trez’} s’occupe de tout ce qui est budget, subvention, comptabilité. Il est responsable de la constante liquidité du club, fonction qui comprend la gestion des relations avec le trésorier du BDE et la banque, ainsi que la gestion des flux et transactions (espèces, chèques). Il dispose de la Golden Card du KI et d’un chéquier. Son rôle (ou plutôt sa carte) est essentiel(le) pour les activités du KI: de l’achat du buffet pour les LAN à la subvention des disques durs.

Dès sa prise de fonction, le trésorier est responsabilisé et doit établir avec le reste du KI le budget du mandat à venir. Il s’assure ensuite tout au long de l’année du bon état des comptes et de la régularité des opérations, tout en se rendant fréquemment à l’agence pour y faire dépôt d’espèces ou de chèques. Ce suivi est très important afin qu’à tout instant le club puisse organiser ses différentes activités.

\sectionenum{Respos}

\subsectionenum{Respo LAN}

Tu es \emph{responsable LAN}. Tu t’occupes de l’événement le plus important du KI.

\subsectionenum{Tech}

\emph{Les respos tech} sont l’épine dorsale du KI. Chaque respo s’occupe d’une facette bien particulière, que ce soit les réparations Windows, Mac, Linux, hardware... Il n’y a pas besoin d’être calé dans dans tout ce qui est info, mais il faut juste un domaine particulier où tu penses pouvoir apporter quelque chose de particulier au KI.

\subsectionenum{Respos Web \& Hébergement}

\emph{Le respo web et hébergement} épaule le Prez’ Tech dans la gestion des services et équipements du KI en relation avec le réseau et internet. Il est responsable des sites internet du KI, de maintenir l’ordre sur l’hébergement web OVH et son pendant sur le VPS avec son installation docker. Il est chargé de traiter les demandes d’hébergement de la part des clubs, associations, élèves et enseignants des Ponts, de configurer ces services d’hébergement, de leur renouvellement au moment des passations, de maintenir à jour les fiches d’hébergement et du
respect de la charte d’hébergement.

\subsectionenum{Respo uPont}

\emph{Le respo uPont} est chargé d’assurer la continuité du bon fonctionnement de uPont. Il est chargé de précréer les comptes de tous les nouveaux élèves avant l’amphi KI de la rentrée à partir de fichiers excel demandés à la DSI. Pour ce faire il doit s’assurer que le nouveau groupe de promotion est fermé (et pas secret), il doit générer un token Facebook pour l’API Graph valide 3 mois et le remplacer dans la app/config/parameter.yml du back Symfony.

Il est responsable des déploiements de uPont, de la prise en compte de tous les feedbacks uP et du maintien du Trello de dev’ uPont.

\subsectionenum{Respo étranger​}

\emph{Le respo étranger} est sur le groupe des étudiants internationaux et est là pour aider les âmes perdues dans un système qu’elles prennent en cours de route.

\sectionenum{Les modérateurs}

\emph{Les Modérateurs} sont les responsables de la limitation du spam dans les listes gérées par le KI, au nombre de quatre cinq histoire de se répartir le boulot. Il faut faire preuve de rigueur, de régularité et faire passer les messages importants en refusant les offres bidon. Il ne faut que du bon sens et une connection internet.

\Footer{\today}

\end{document}
