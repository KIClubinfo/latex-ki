% Partie l'essentiel du Guide de connexion internet des Ponts, KI020, août 2018

% The ki020 needs XeLaTeX to use the specified fonts
% The minted package requires the python package Pigments.
% Minted makes calls to the outside world (Pigments), that's why the source must be compiled from terminal with the -shell-escape argument.

% sudo pip3 install Pigments
% In the guide-connexion's directory :
% xelatex -shell-escape guide_connexion_internet_ponts.tex
% xelatex -shell-escape guide_connexion_internet_ponts.tex
% (twice to generate the table of contents

\documentclass{../templates/enpc-ki/ki020}

\usepackage{lipsum}
\usepackage[hidelinks]{hyperref}
\usepackage{minted}
\usepackage{graphicx}
\usepackage{float}
\usepackage{wrapfig}

\graphicspath{{images/}}
\pagestyle{empty}

\begin{document}

\begin{tikzpicture}[remember picture,overlay,baseline=0pt]
\fill[primary]
  ([yshift=-4.4cm]current page.north west) --
  ([xshift=-4.62cm,yshift=-2.85cm]current page.north east) --
  ([yshift=-5.27cm]current page.north west) -- cycle;
\fill[header]
  (current page.north west) -- (current page.north east) --
  ([yshift=-2.45cm]current page.north east) --
  ([xshift=-4.62cm,yshift=-2.89cm]current page.north east) --
  ([yshift=-4.44cm]current page.north west) -- cycle;
  \node[anchor=base west, white,font=\Huge\titlefont,xshift=4.5cm,yshift=-2cm] at (current page.north west){La connexion Internet};
  \node[anchor=north west,xshift=0.5cm,yshift=-0.2cm] at (current page.north west)
    {\includegraphics[height=3cm]{../../logos/Logo_fond_gris_blanchi.png}};
\end{tikzpicture}

\vspace{4.3cm}

    \begin{kiframe}
      \subsection{Dans les résidences} %TODO refaire subsubsection
        Une seule des deux prises RJ45 est reliée à Internet, la bonne est en général la plus proche du bureau.

        Les routeurs vendus par le KI doivent être reliés au mur grâce à la prise \emph{bleue} (port WAN). En cas de problème, ou si tu comptes utiliser ton propre routeur, n'hésite pas a nous demander.

        \begin{flushleft}
          \emph{Proxy des résidences :}
          \begin{itemize}
            \item Adresse : etuproxy.enpc.fr
            \item Port : 3128
            \item Login : identifiants DSI (prénom.nom@enpc.fr)
          \end{itemize}
        \end{flushleft}

      \subsection{Internet aux Ponts}
        \begin{flushleft}
          \emph{Wifi des Ponts :}
          \begin{itemize}
            \item SSID : eduroam
            \item Login : identifiants DSI (prénom.nom@enpc.fr)
            \item \emph{Bien désactiver la vérification du certificat}
          \end{itemize}
        \end{flushleft}

\subsection{Guide complet}
  \begin{flushleft}
    \emph{Le guide de connexion complet est disponible sur enpc.org/guide}
  \end{flushleft}
      \end{kiframe}

\end{document}
